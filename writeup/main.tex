\documentclass[12pt]{scrartcl}
\usepackage[superscript,biblabel]{cite}
\usepackage{graphicx}
\usepackage{hyperref}
\usepackage{fullpage}
\usepackage{setspace}
\usepackage{csquotes}
\usepackage{listings}
\usepackage{indentfirst}
\usepackage[T1]{fontenc}
\usepackage{titling}

\setlength{\droptitle}{-6em}
\doublespacing
\hypersetup{%
  colorlinks=true,% hyperlinks will be coloured
  linkcolor=blue,% hyperlink text will be green
}

\begin{document}

	\title{Distributed Systems and Algorithms\\Project 2}
	\author{Theo Browne and Paul Revere}
	\date{\vspace{-5ex}}

	\maketitle



	\section*{General Overview}

	To demonstrate our understanding of Zookeeper and similar distributed systems, we have built a distributed system in Python in the style of Zookeeper. A given number of servers are running at any time with identical file systems. Changes to the filesystem are requested by a client, forwarded from the server to the "leader", and the "leader" approves the changes and broadcasts them to every server. The servers are "followers", and the lead server has a separate "leader" thread. This "leader" thread runs alongside the follower thread on the lead server, so changes approved by "leader" are sent to every follower, even itself. 

	When a "transaction\_request" is made by a follower to the leader, the leader sends out a "transaction\_proposal" to all followers. If the majority of the servers reply with a "transaction\_acknowledge", the transaction is sent to every follower as a "transaction\_commit", completing a two-phase commit.

	When a "commit\_change" message is received by a "follower", the change is committed to the file system and written to the hisory (also worth noting that the history is logged as a local file "server\_(server\_num).history"). 

	Upon coming online, a given server immediately launches an election. The server will attempt to connect to every address in the config, storing the sockets that are successfully connected. For each successful connection, an "election" message is sent. Once a reply is received from every "election" message (if the server fails after receiving the message, it will reply with an empty string), it knows the election is over. Any server with an id higher than the server that initialized the election will start an election of it's own. The first server to not receive a "higher\_id" in it's election\_responses becomes leader. 

	When a server goes down, anything connected to it will receive a blank string upon an attempt to read. When a disconnection occurs, the socket it occured on is removed from the list of active sockets. If the disconnection was the leader, a new election is held.

	If a new server comes online when a file system is already established, the election winner will likely be the server that was already leading before the new server came online. When the election is completed, every non-leader node wipes it's history and file system. When a new leader is initialized, it first sends a 'leader' message to let clients know that it is a leader, and follows it up with all of the 'commit' messages in it's history, bringing all followers up to speed with the latest state of the file system.

	\section*{Network configuration}

	To run the server, the following command is run:
	\begin{lstlisting}
	python server.py (server_number) (path_to_config) [Optional: path_to_history]
	\end{lstlisting}

	The config file is a list of server numbers, IPs, and ports to connect to. The connection information corresponding to the argument given in server\_number is assigned to the sockets in that server's listeners.

	There are 3 ports for each IP. One for the client, one for server-to-server connections, and one for Leader-to-Follower connections. This allows us to reliably know where new messages are coming from.

	To detect the status of a given socket, we use the C library Select. Select takes a list of sockets and returns which ones are ready to be read, written to, or have errored out. The "errored out" functionality of Select is unreliable at best as any socket that disconnects without a proper "shutdown" is readable (but only sends a blank string).

	If the path\_to\_history argument is provided, the server assumes it is rebuilding from a logged history before reconnecting to the network. It builds it's filesystem up based on the provided history before initiating an election.

	The client connects to any given server through it's client\_listener port. It only connects temporarily while sending a message, and only maintains the connection until a reply is received. This allows for a theoretical "unlimited" amount of clients connected.

	% To assure no collisions occur due to multiple nodes accessing/holding the same file, request queues are used. A single FIFO request queue exists within each node, tracking the requests that have been made and completing them one at a time. 

	% A config file is used to configure each individual node, each of which is an individual Python instance. The configuration file lists the node name, IP address, and port for the node being initialized and all of its neighbors.

	% There are two main threads in each Python instance, one for listening to requests, and one for processing them. Whenever the listening thread receives a request (TCP requests through Python SocketServer), the request is deserialized and added to the requests queue. The processing thread is constantly cycling through the queue, completing tasks when it can and adding them back to the end of the queue when it is waiting for a return. 

	% \section*{Function Descriptions}

	% The \textit{create} function adds a representation of the "file" to a local Python dictionary, titled \textit{files}. A new, empty file is created by calling \textit{files[filename] = ``''}. This creation is broadcast through the network by \textit{broadcast\_after\_create}, allowing for every node to update their \textit{token\_directions\{\}} map. 

	% The \textit{delete} function obtains and deletes the file (and token) for a given filename. If the node calling \textit{delete} does not currently have the \textit{filename} in \textit{files}, it references \textit{token\_directions} and makes a request in the direction for the given \textit{filename}. Upon receiving a return with a file, it is automatically added to \textit{files}, so finalizing the deletion is as simple as removing the element from \textit{files} once the request and return are complete. 

	% The \textit{read} function works identically to delete, but ends by printing the content of the file instead of deleting the file from \textit{files}.

	% \textit{Append} also works in the same pattern as \textit{delete} and \textit{read}, requesting the file (token) if it is not already held in \textit{files}. Once the file is obtained, the requested append is made with a simple plus-equals, formatted as \textit{files[filename] += text\_to\_append}.

\end{document}